%---------------------------------------------------------------------
%
%                      agradecimientos.tex
%
%---------------------------------------------------------------------
%
% agradecimientos.tex
% Copyright 2009 Marco Antonio Gomez-Martin, Pedro Pablo Gomez-Martin
%
% This file belongs to the TeXiS manual, a LaTeX template for writting
% Thesis and other documents. The complete last TeXiS package can
% be obtained from http://gaia.fdi.ucm.es/projects/texis/
%
% Although the TeXiS template itself is distributed under the 
% conditions of the LaTeX Project Public License
% (http://www.latex-project.org/lppl.txt), the manual content
% uses the CC-BY-SA license that stays that you are free:
%
%    - to share & to copy, distribute and transmit the work
%    - to remix and to adapt the work
%
% under the following conditions:
%
%    - Attribution: you must attribute the work in the manner
%      specified by the author or licensor (but not in any way that
%      suggests that they endorse you or your use of the work).
%    - Share Alike: if you alter, transform, or build upon this
%      work, you may distribute the resulting work only under the
%      same, similar or a compatible license.
%
% The complete license is available in
% http://creativecommons.org/licenses/by-sa/3.0/legalcode
%
%---------------------------------------------------------------------
%
% Contiene la p�gina de agradecimientos.
%
% Se crea como un cap�tulo sin numeraci�n.
%
%---------------------------------------------------------------------

\chapter{Agradecimientos}

\cabeceraEspecial{Agradecimientos}

Queremos dar las gracias en primer lugar a nuestras directoras, Virginia y Raquel, por su esfuerzo y dedicaci�n a este trabajo. Ha sido un aut�ntico placer haber tenido la oportunidad de trabajar con dos profesoras tan buenas tanto en su trabajo como a nivel personal. Nos han aportado mucho conocimiento y nos han ayudado a sacar lo mejor de nosotras.

\hfill \break
En segundo lugar, queremos dar las gracias a la asociaci�n Asperger Madrid por toda la ayuda que nos han ofrecido a lo largo del trabajo. Hemos aprendido mucho gracias a todos ellos y nos ha encantado poder colaborar con la asociaci�n ya que nos ha permitido ver la importancia que tiene trabajar en el desarrollo de herramientas para el an�lisis emocional.

\hfill \break
Por �ltimo, por supuesto agradecer a nuestras familias y amigos el apoyo que nos han dado a lo largo de este �ltimo a�o. 

\endinput
% Variable local para emacs, para  que encuentre el fichero maestro de
% compilaci�n y funcionen mejor algunas teclas r�pidas de AucTeX
%%%
%%% Local Variables:
%%% mode: latex
%%% TeX-master: "../Tesis.tex"
%%% End:
