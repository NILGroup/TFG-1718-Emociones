%---------------------------------------------------------------------
%
%                      resumen.tex
%
%---------------------------------------------------------------------
%
% Contiene el cap�tulo del resumen.
%
% Se crea como un cap�tulo sin numeraci�n.
%
%---------------------------------------------------------------------

\chapter{Resumen/Summary}
\cabeceraEspecial{Resumen/Summary}

En la actualidad, gracias a Internet, tenemos acceso f�cil y r�pido a una cantidad enorme de informaci�n. Lo �nico que necesitamos es interpretar esta informaci�n correctamente para saber si es �til para nosostros o no y de qu� manera. En este punto surge un problema porque no todas las personas tienen la misma facilidad para interpretar un texto, por ejemplo, la falta de empat�a emocional en las personas con Trastornos del Espectro Autista (TEA) les impide reconocer el contenido afectivo de un texto lo que puede llegar a alterar en gran medida su significado. Algo similar ocurre cuando estas personas quieren escribir algo en la red: la dificultad para expresar sus emociones puede provocar malentendidos. Es necesario desarrollar herramientas que permitan la integraci�n digital de personas con este tipo de discapacidad. 

La computaci�n afectiva ha ayudado a conseguir grandes avances en este �mbito llegando a permitir identificar el estado emocional del sujeto a trav�s de diferentes fuentes: voz, texto, expresiones faciales... En el caso de un texto se pueden reconocer las emociones contenidas en este mediante diversas t�cnicas basadas en analizar aspectos de su sint�xis y sem�ntica: el tipo de frase que lo forman, las palabras que contiene, el contexto que se puede deducir... Un ejemplo es el m�todo \textit{keyword spotting} que puede ayudarse de diccionarios afectivos, que contienen una serie de palabras acompa�adas de sus categor�as y/o dimensiones emocionales, para detectar palabras clave en el texto a partir de las cuales se puede realizar el an�lisis afectivo.

Este trabajo emplear� precisamente este �ltimo m�todo para desarrollar una aplicaci�n web que sea capaz de detectar autom�ticamente el grado de carga afectiva que tiene un texto. Para ello nos ayudaremos de un diccionario afectivo que nos permitir� marcar las palabras que aparecen con un grado de cada una de las cinco categor�as emocionales con las que hemos decidido trabajar: tristeza, miedo, alegr�a, enfado y asco. El resultado se mostrar� en la interfaz de forma simple y gr�fica, mediante emoticonos y colores.

Una vez finalizado, el trabajo fue sometido a una evaluaci�n donde se realizaron pruebas con diferentes tipos de textos escogidos por nosotras y a dos evaluaciones con usuarios finales y expertos para probar tanto la funcionalidad como la interfaz.

\newpage

Nowadays, thanks to the Internet, we have quick and easy access to a huge amount of information. All we need is to interpret this information correctly to know if it is useful for us or not and how if so. At this point a problem arises because not everyone has the same easy to interpret a text, for example, the lack of emotional empathy in people with Autism Spectrum Disorders (ASD) impedes them to recognize the affective content in a text which can greatly alter its meaning. Something similar happens when these people want to write something on the net: the difficulties they have when expressing their emotions can cause misunderstandings. It is necessary to develop tools which could allow the digital integration of people with this type of disability.

Affective computing has helped to achieve great advances in this area, allowing us to identify the emotional state of a person through different sources: voice, text, facial expressions... In a text, the emotions it contains can by recognized through various techniques based on analyzing aspects of text's syntax and semantic: the type of phrases it is formed by, the words which it contains, the context that can be inferred... An example is the method \textit{keyword spotting} that can use an affective dictionary to detect key words in the text from which the affective analysis can be carried out. An affective dictionary contains a series of words accompanied by their emotional categories or their emotional dimensions. 

This project will use precisely this last method to develop a web application that is capable of automatically detect the degree of affective weight of a text. For this, we will use an affective dictionary that will allow us to mark the words that appear with a grade of each of the five emotional categories with which we have decided to work: sadness, fear, joy, anger and disgust. The result will be displayed in the interface in a simple and graphic way, using emoticons and colors.

Once completed, the project was submitted to an evaluation where tests were performed with different types of texts chosen by us and also to other two evaluations with end-users and experts to test both the functionality and the interface.

\endinput
% Variable local para emacs, para  que encuentre el fichero maestro de
% compilaci�n y funcionen mejor algunas teclas r�pidas de AucTeX
%%%
%%% Local Variables:
%%% mode: latex
%%% TeX-master: "../Tesis.tex"
%%% End:
