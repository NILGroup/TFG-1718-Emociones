%---------------------------------------------------------------------
%
%                      resumen.tex
%
%---------------------------------------------------------------------
%
% Contiene el cap�tulo del resumen.
%
% Se crea como un cap�tulo sin numeraci�n.
%
%---------------------------------------------------------------------

\chapter*{Resumen}
\cabeceraEspecial{Resumen}

En la actualidad, gracias a Internet, tenemos acceso f�cil y r�pido a una cantidad enorme de informaci�n. Lo �nico que necesitamos es interpretar esta informaci�n correctamente para saber si es �til para nosostros o no y de qu� manera. En este punto surge un problema porque no todas las personas tienen la misma facilidad para interpretar un texto, por ejemplo, la falta de empat�a emocional en las personas con Trastornos del Espectro Autista (TEA) les impide reconocer el contenido afectivo de un texto lo que puede llegar a alterar en gran medida su significado. Algo similar ocurre cuando estas personas quieren escribir algo en la red: la dificultad para expresar sus emociones puede provocar malentendidos. Es necesario desarrollar herramientas que permitan la integraci�n digital de personas con este tipo de discapacidad. 
 
La computaci�n afectiva ha ayudado a conseguir grandes avances en este �mbito llegando a permitir identificar el estado emocional del sujeto a trav�s de diferentes fuentes: voz, texto, expresiones faciales... En el caso de un texto se pueden reconocer las emociones contenidas en este mediante diversas t�cnicas basadas en analizar aspectos de su sint�xis y sem�ntica: el tipo de frase que lo forman, las palabras que contiene, el contexto que se puede deducir... La mayor�a de estos avances se han realizado para el ingl�s, todav�a queda mucho por hacer en castellano.

Este trabajo busca desarrollar una aplicaci�n web que sea capaz de detectar autom�ticamente el grado de carga afectiva que tiene un texto. Representaremos la carga afectiva de un texto mediante las categor�as emocionales que representan las cinco emociones b�sicas: tristeza, miedo, alegr�a, enfado y asco. El resultado se mostrar� en la interfaz de forma simple y gr�fica, mediante emoticonos y colores.

Una vez finalizado, el trabajo fue sometido a una evaluaci�n donde se realizaron pruebas internas con diferentes tipos de textos escogidos por nosotras y a una evaluaci�n con usuarios finales para probar tanto la funcionalidad como la interfaz. El resultado de estas evaluaciones indica que a�n queda trabajo por hacer pero hemos establecido una base muy s�lida para el trabajo futuro.

\endinput
% Variable local para emacs, para  que encuentre el fichero maestro de
% compilaci�n y funcionen mejor algunas teclas r�pidas de AucTeX
%%%
%%% Local Variables:
%%% mode: latex
%%% TeX-master: "../Tesis.tex"
%%% End:
