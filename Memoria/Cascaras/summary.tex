%---------------------------------------------------------------------
%
%                      resumen.tex
%
%---------------------------------------------------------------------
%
% Contiene el cap�tulo del resumen.
%
% Se crea como un cap�tulo sin numeraci�n.
%
%---------------------------------------------------------------------

\chapter*{Summary}
\cabeceraEspecial{Summary}

Nowadays, thanks to the Internet, we have quick and easy access to a huge amount of information. All we need is to interpret this information correctly to know if it is useful for us. At this point a problem arises because not everyone has the same easy to interpret a text, for example, the lack of emotional empathy in people with Autism Spectrum Disorders (ASD) impedes them to recognize the affective content in a text which can greatly alter its meaning. Something similar happens when these people want to write something on the network: the difficulties they have when expressing their emotions can cause misunderstandings. It is necessary to develop tools which could allow the digital integration of people with this type of disability.

Affective computing has helped to achieve great advances in this area, allowing us to identify the emotional state of a person through different sources: voice, text, facial expressions... In a text, the emotions it contains can by recognized through various techniques based on analyzing aspects of text's syntax and semantic: the type of phrases it is formed by, the words which it contains, the context that can be inferred... Most of these advances have been made for the English language, but there is still much to do in Spanish.

This project aims to develop a web application that is capable of automatically detect the degree of the emotions of a text. We will represent the affective weight of a text through the emotional categories that represent the five basic emotions: sadness, fear, joy, anger and disgust. The result will be displayed in the interface in a simple and graphic way, using emoticons and colors.

Once completed, the project was evaluated. Different types of texts were chosen by us to evaluate the accuracy of our emotional annotation and an evaluation with end-users was performed to test both the functionality and the interface. The result of these evaluations indicates that there is still work to be done but we have established a very solid foundation for future work.

\endinput
% Variable local para emacs, para  que encuentre el fichero maestro de
% compilaci�n y funcionen mejor algunas teclas r�pidas de AucTeX
%%%
%%% Local Variables:
%%% mode: latex
%%% TeX-master: "../Tesis.tex"
%%% End:

