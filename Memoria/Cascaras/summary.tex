%---------------------------------------------------------------------
%
%                      resumen.tex
%
%---------------------------------------------------------------------
%
% Contiene el cap�tulo del resumen.
%
% Se crea como un cap�tulo sin numeraci�n.
%
%---------------------------------------------------------------------

\chapter*{Summary}
\cabeceraEspecial{Summary}

Nowadays, thanks to the Internet, we have quick and easy access to a huge amount of information. All we need is to interpret this information correctly to know if it is useful for us or not and how if so. At this point a problem arises because not everyone has the same easy to interpret a text, for example, the lack of emotional empathy in people with Autism Spectrum Disorders (ASD) impedes them to recognize the affective content in a text which can greatly alter its meaning. Something similar happens when these people want to write something on the net: the difficulties they have when expressing their emotions can cause misunderstandings. It is necessary to develop tools which could allow the digital integration of people with this type of disability.

Affective computing has helped to achieve great advances in this area, allowing us to identify the emotional state of a person through different sources: voice, text, facial expressions... In a text, the emotions it contains can by recognized through various techniques based on analyzing aspects of text's syntax and semantic: the type of phrases it is formed by, the words which it contains, the context that can be inferred... An example is the method \textit{keyword spotting} that can use an affective dictionary to detect key words in the text from which the affective analysis can be carried out. An affective dictionary contains a series of words accompanied by their emotional categories or their emotional dimensions. 

This project will use precisely this last method to develop a web application that is capable of automatically detect the degree of affective weight of a text. For this, we will use an affective dictionary that will allow us to mark the words that appear with a grade of each of the five emotional categories with which we have decided to work: sadness, fear, joy, anger and disgust. The result will be displayed in the interface in a simple and graphic way, using emoticons and colors.

Once completed, the project was submitted to an evaluation where tests were performed with different types of texts chosen by us and also to other two evaluations with end-users and experts to test both the functionality and the interface.

\endinput
% Variable local para emacs, para  que encuentre el fichero maestro de
% compilaci�n y funcionen mejor algunas teclas r�pidas de AucTeX
%%%
%%% Local Variables:
%%% mode: latex
%%% TeX-master: "../Tesis.tex"
%%% End:

