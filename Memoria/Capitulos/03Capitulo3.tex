%---------------------------------------------------------------------
%
%                          Cap�tulo 1
%
%---------------------------------------------------------------------

\chapter{An�lis del Contenido Afectivo de un Texto}

\begin{FraseCelebre}
\begin{Frase}
...
\end{Frase}
\begin{Fuente}
...
\end{Fuente}
\end{FraseCelebre}

\begin{resumen}
En este cap�tuo se explicar�n los servicios web que se han desarrollado con el fin de analizar el contenido afectivo de un texto.
Primero, en la secci�n 3.1, se presentar�n los servicios orientados a identificar la emoci�n predominante en una palabra concreta, tanto la consensuada como la mayoritaria para quedarnos con la opci�n m�s correcta. Despu�s, en la secci�n 3.2, se explica como el an�lisis de la palabra se aplica a determinar la emoci�n de una frase entera para finalmente, en la secci�n 3.3, aplicar todo al an�lis de todo un texto.
\end{resumen}


%-------------------------------------------------------------------
\section{�nalis afectivo de una palabra}
%-------------------------------------------------------------------
\label{cap3:sec:palabra}

 Como ya se coment� en la secci�n 2.1.2, para el desarrollo de este trabajo se utilizar� un diccionario afectivo en espa�ol. Este diccionario, inicialmente en formato Excel, ha sido convertido a un fichero CSV para que resulte m�s sencillo obtener los datos. 

Teniendo en cuenta que el trabajo est� orientado a ser utilizado mediante servicios web lo primero que hay que hacer es subir la informaci�n que tenemos, todas las palabras con sus respectivos grados de certeza para cada emoci�n, a un servidor sobre el que se realizar�n las distintas peticiones. El framework que hemos utilizado para todo el proceso es \textbf{Django}. \textbf{Django} nos permite modelar una palabra para almacenarla en el servidor y programar una serie de clases "`vista"' que nos permiten acceder a la lista de palabras.
\begin{itemize}
	\item \textbf{Modelo: } Los campos que tiene el modelo para cada palabra son la propia palabra, el porcentaje de una determinada emoci�n que expresan (para las seis emociones que consideramos en nuestro diccionario) y la emoci�n cuyo porcentaje destaca. Cada una de las palabras que tenemos recogidas se encapsulan en este modelo para ser almacenadas en el servidor.
	\item \textbf{Consultas: } Para realizar las difetentes consultas sobre las palabras disponibles se crean una serie de clases que implementan los diferentes m�todos de un servicio web REST t�pico: \textbf{GET, POST, DELETE}. Esencialmente \textbf{GET}, ya que es el que devuelve informaci�n sobre la palabra o alguno de sus campos. Cada una de las diferentes clases o "`vistas"' aportan una forma diferente de acceder a la informaci�n: acceso a toda la lista, a una palabra concreta, a un campo de una palabra concreta, etc...
\end{itemize}
Una vez que se tiene el servidor y todo el c�digo necesario para el funcionamiento de los servicios web se procede a subir las palabras que recoge nuestro diccionario. Para ello se ha desarrollado un programa en \textit{Python} cuya funci�n es leer el fichero CSV, interpretar cada una de sus l�neas y subir la informaci�n que obtiene al servidor. Una vez que todas las palabras est�n subidas ya se pueden realizar las consultas necesarias.

Inicialmente se ha implementado un servicio web, tambi�n en \textit{Python}, que dada una palabra devuelve la informaci�n respectiva a los porcentajes de cada emoci�n que posee, mediante una petici�n \textbf{GET} al servidor, y los muestra al usuario.

%-------------------------------------------------------------------
\section{�nalis afectivo de una frase}
%-------------------------------------------------------------------
\label{cap3:sec:frase}

...

%-------------------------------------------------------------------
\section{�nalis afectivo de un texto}
%-------------------------------------------------------------------
\label{cap3:sec:texto}

...

%-------------------------------------------------------------------
\section*{\NotasBibliograficas}
%-------------------------------------------------------------------
\TocNotasBibliograficas

Citamos algo para que aparezca en la bibliograf�a\ldots
\citep{ldesc2e}

\medskip

Y tambi�n ponemos el acr�nimo \ac{CVS} para que no cruja.

Ten en cuenta que si no quieres acr�nimos (o no quieres que te falle la compilaci�n en ``release'' mientras no tengas ninguno) basta con que no definas la constante \verb+\acronimosEnRelease+ (en \texttt{config.tex}).


%-------------------------------------------------------------------
\section*{\ProximoCapitulo}
%-------------------------------------------------------------------
\TocProximoCapitulo

...

% Variable local para emacs, para  que encuentre el fichero maestro de
% compilaci�n y funcionen mejor algunas teclas r�pidas de AucTeX
%%%
%%% Local Variables:
%%% mode: latex
%%% TeX-master: "../Tesis.tex"
%%% End:
