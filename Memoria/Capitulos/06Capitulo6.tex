%---------------------------------------------------------------------
%
%                          Cap�tulo 6
%
%---------------------------------------------------------------------

\chapter{Desarrollo de la aplicaci�n web}

\begin{resumen}
En este cap�tulo se explicar� el proceso de desarrollo de la aplicaci�n web. 
\end{resumen}

%-------------------------------------------------------------------
\section{Despliegue}
%-------------------------------------------------------------------
\label{cap6:sec:despliegue}

El primer paso es desplegar los servicios web en el servidor que nos ha proporcionado la universidad. Desplegaremos la aplicaci�n desarrollada en Django en el apache del servidor usando mod\_wsgi, un m�dulo de apache que le permite hospedar aplicaciones web desarrolladas en Python. 

Comenzamos instalando apache y el m�dulo en el servidor (Figura \ref{fig:install}) y, una vez que ambos est�n instalados, habilitamos el m�dulo (Figura \ref{fig:modulo}) y reiniciamos apache (Figura \ref{fig:restart}) para que tenga efecto. Para que nuestra aplicaci�n quede alojada en el apache hay que crear un virtualhost para ella (un dominio que apunte al directorio donde est� nuestro proyecto). El virtualhost se configura creando un fichero con extensi�n .conf cuyo contenido se muestra en la Figura \ref{fig:conf}.

\figura{Bitmap/Capitulo6/conf}{width=.6\textwidth}{fig:conf}{Contenido del archivo .conf}
\figura{Bitmap/Capitulo6/install}{width=.6\textwidth}{fig:install}{Comando para la instalaci�n}
\figura{Bitmap/Capitulo6/modulo}{width=.6\textwidth}{fig:modulo}{Comando para la activaci�n de mod\_wsgi}
\figura{Bitmap/Capitulo6/restart}{width=.6\textwidth}{fig:restart}{Comando para el reinicio de apache}
\figura{Bitmap/Capitulo6/activate}{width=.6\textwidth}{fig:activate}{Comando para activar el virtualhost}


Cuando hemos terminado con la configuraci�n del virtualhost s�lo queda activarlo (Figura \ref{fig:activate}) y volver a reiniciar el apache.

%-------------------------------------------------------------------
\section{Desarrollo de la interfaz}
%-------------------------------------------------------------------
\label{cap6:sec:interfaz}

La interfaz se desarrolla siguiendo el prototipo final obtenido en la secci�n 5.3.

%-------------------------------------------------------------------
\section{Integraci�n}
%-------------------------------------------------------------------
\label{cap6:sec:integracion}

Integrar la interfaz con la parte funcional.