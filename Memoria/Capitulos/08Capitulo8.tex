
%---------------------------------------------------------------------
%
%                          Capitulo 8
%
%---------------------------------------------------------------------

\chapter{Trabajo Individual}

\begin{resumen}
En este cap�tulo se presentar�n el trabajo individual realizado por cada una de nosotras durante el proyecto, as� como los conocimientos que hemos y adquirido a lo largo de este.
\end{resumen}

% Est�n provisionalmente por orden alfab�tico, cuando hablemos de qu� poner los podemos cambiar de orden para que cuadre mejor.

%--------
\section{Gema}
%--------
\label{cap9:sec:gema}

Lo primero que hicimos al comenzar el Trabajo de Fin de Grado fue investigar el estado del arte y empezar a documentarlo en la memoria. Yo me encargu� de investigar los distintos diccionarios utilizados para marcar textos y el funcionamiento de los servicios web, centr�ndome en los REST ya que eran los que �bamos a usar. Adem�s de a�adir el primer cap�tulo a la memoria tuvimos que desarrollar los primeros servicios web (obtener la lista de palabras y obtener una palabra). Una vez que Paloma hizo el modelo y ten�amos los datos en el servidor program� ambas vistas y lo document� en la memoria.

%--------
\section{Paloma}
%--------
\label{cap9:sec:paloma}

Toda comenzamos redactando el cap�tulo de la memoria relativo al estado del arte investigando una serie de puntos que nos marcaron las profesoras y nos repartimos entre las tres. Mi aportaci�n fue investigar el funcionamiento del diccionario con el que �bamos a trabajar y la metodolog�a que �bamos a utilizar (scrum). Para empezar a desarrollar los servicios web necesit�bamos un modelo que utilizase la base de datos de Django para representar las palabras, yo me encargu� de desarrollar este modelo y documentarlo en la memoria comenzando el cap�tulo 3 de esta. Cuando Elena termin� el servicio web para obtener los porcentajes, al igual que hizo ella para probar los servicios que hizo Gema, hice un fichero para comprobar que el comportamiento era el esperado.

%--------
\section{Elena}
%--------
\label{cap9:sec:elena}

Al igual que mis compa�eras comenc� el proyecto investigando una parte del estado del arte: la computaci�n afectiva, en qu� consiste y sus aplicaciones, y el modelo inform�tico de integraci�n continua. Para comenzar con la parte t�cnica lo primero que ten�amos que hacer era conseguir tener los datos del diccionario en el servidor, yo me encargu� de crear un fichero en Python que leyese el CSV que conten�a las palabras del diccionario y las subiese a la base de datos de Django. Una vez que mi compa�era Gema termin� los servicios web yo hice un fichero de pruebas para comparar la salida esperada con lo que devolv�a el servidor. Adem�s, document� lo que hab�a hecho en la memoria. Cuando ya ten�amos la estructura montada y funcionaban los servicios b�sicos ten�amos que empezar a desarrollar los que �bamos a usar. El primero que hicimos fue el que nos permit�a obtener los porcentajes de cada emoci�n.
