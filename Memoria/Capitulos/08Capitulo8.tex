
%---------------------------------------------------------------------
%
%                          Capitulo 7
%
%---------------------------------------------------------------------

\chapter{Conocimientos Aplicados}

\begin{resumen}
En este cap�tulo se explica la manera en la que se aplican los conocimientos adquiridos a lo largo de la carrera.
\end{resumen}


A lo largo de la carrera hemos tenido la oportunidad de estudiar una gran serie de asignaturas que nos han sido de gran utilidad a lo largo del desarrollo de este Trabajo de Fin de Grado.

Para empezar, han sido fundamentales las asignaturas de Fundamentos de la Programaci�n y Tecnolog�a de la Programaci�n, ya que nos han ayudado a adquirir una forma de pensar estructurada y eficiente a la hora de programar, adem�s de ense�arnos lenguajes tan potentes con C++ o Java respectivamente. Conocer estos lenguajes nos ha hecho m�s f�cil aprender Python, un lenguaje al que no nos hab�amos enfrentado antes. La asignatura de Estructura de Datos y Algoritmos nos ha permitido ampliar nuestros conocimientos como programadoras aport�ndonos herramientas para poder hacer nuestros programas lo m�s eficientes posible lo que se puede ver reflejado en el c�digo del proyecto.

Al ser este un proyecto tan grande y realizado a lo largo de todo un a�o hemos tenido que ser capaces de gestionar todo lo relacionado con �l: equipo, planificaci�n, pruebas... Todos estos conocimientos los hemos obtenido en la asignatura de Ingenier�a del Software donde ya tuvimos la oportunidad de llevar a cabo un proyecto anual gestionando cada parte de este siguiendo las metodolog�as que m�s se adaptaban al proyecto en cuesti�n. En esta asignatura tambi�n aprendimos los diferentes tipos de pruebas que existen para software (caja blanca, caja negra..) y aprendimos a aplicarlas. Otras asignaturas que nos han ayudado a dise�ar y realizar las pruebas son Auditor�a Inform�tica y Evaluaci�n de Configuraciones que nos han permitido poner a prueba tanto los servicios web como la aplicaci�n desde distintos puntos de vista.

Otro m�dulo muy importante de asignaturas es el compuesto por las asignaturas relacionadas con desarrollo web: Aplicaciones Web, Ampliaci�n de Bases de Datos e Ingenier�a Web. Estas asignaturas nos han aportado los conocimientos necesarios de HTML, CSS y JavaScript para poder ser capaces de desarrollar la aplicaci�n web as� como las herramientas necesarias para poder integrar esta aplicaci�n con los servicios web mediante el uso de JQuery.

Antes de desarrollar la aplicaci�n web tuvimos que dise�ar su interfaz para lo que nos result� realmente �til la asignatura de Desarrollo de Sistemas Interactivos donde aprendimos el proceso a seguir a la hora de desarrollar una aplicaci�n. Aprendimos a dise�ar prototipos como los que hemos dise�ado para este proyecto y realizar evaluaciones con usuarios para probar tanto los prototipos como la aplicaci�n, lo que tambi�n ha sido aplicado en nuestro proyecto gracias a la gente de la Asociaci�n de Asperger Madrid.

En cuanto al desarrollo de los servicios web, nos ha ayudado la asignatura de Bases de Datos para poder crear la base de datos que forma el modelo de Django y realizar las distintas consultas necesarias sobre ella. Si bien estas consultas no se hacen de forma directa por nosotras nos ayuda entender como funciona la base de datos y como se almacenan y obtienen los datos.

Para tener todo disponible en Internet necesit�bamos utilizar un servidor que nos fue proporcionado por nuestras directoras del TFG. Este servidor ten�a el sistema operativo GNU/Linux que aprendimos a manejar en la asignatura de Sistemas Operativos. Para realizar el despliegue de nuestro proyecto en Apache y gestionar todo lo relacionado con el servidor nos ayudaron los conocimientos adquiridos en las asignaturas relacionadas con redes: Redes, Redes y Seguridad y Ampliaci�n de Redes y Sistemas Operativos.

Por �ltimo, en la asignatura de �tica, Legislaci�n y Profesi�n aprendimos todo lo necesario sobre licencias tanto para proteger nuestro c�digo como para saber utilizar de forma correcta el software libre desarrollado por otras personas.

