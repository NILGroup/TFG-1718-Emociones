%---------------------------------------------------------------------
%
%                          Cap�tulo 1
%
%---------------------------------------------------------------------

\chapter{Introducci�n}

\begin{resumen}
	En este cap�tulo se va a realizar una introducci�n al trabajo que vamos a presentar.
	En primer lugar, en la secci�n 1.1 se explicar� en qu� consiste el trabajo. En la secci�n 1.2 se expondr�n las principales dificultades a las que nos hemos enfrentado y las decisiones que se han tomado para afrontarlas. Por �ltimo, en la secci�n 1.3 se presentar� la estructura del trabajo.

\end{resumen}
 
%-------------------------------------------------------------------
\section{Definici�n}
%-------------------------------------------------------------------
\label{cap1:sec:definicion}

El tabajo se centra en la detecci�n autom�tica del contenido emocional de un texto, una parte de la computaci�n afectiva que permite identificar la carga emocional de un texto. Esto nos va a ayudar a evitar ambig�edades emocionales a la hora de interpretar un texto facilitando su lectura a personas con alg�n d�ficit en la percepci�n de emociones, como aquellas con Trastornos del Espectro Autista.
Se implementar�n servicios web que permitan analizar emocionalmente un texto, identificando las emociones b�sicas que contiene y su intensidad, y hacer m�s explicito el significado emocional para hacer m�s f�cil su comprensi�n a las personas con discapacidades de este tipo.

%-------------------------------------------------------------------
\section{Dificultades}
%-------------------------------------------------------------------
\label{cap1:sec:dificultades}

A lo largo de la realizaci�n del trabajo nos hemos ido encontrando una serie de problemas que se explican a continuaci�n.

\subsection{Gram�tica}
Para simplificar la subida de los datos al servidor, no tener que trabajar con caract�res especiales, hemos eliminado las tildes de las palabras de nuestro diccionario y reemplazado las "`�"' por "`ny"'. Esto no tiene efecto en la interacci�n con los usuarios ya que ellos introducen las palabras como realmente se escriben y se realiza la transformaci�n antes de buscarlas en el servidor. Al no trabajar con palabras con tildes diacr�ticas no se producen ambig�edades a la hora de realizar la b�squeda as� que tampoco afectar� al funcionamiento.

%-------------------------------------------------------------------
\section{Estructura}
%-------------------------------------------------------------------
\label{cap1:sec:estructura}

Este cap�tulo 1 sirve de introducci�n al trabajo definiendo su objetivo, su estructura y los problemas que han aparecido a lo largo de su desarrollo.
El cap�tulo 2 presenta los aspectos m�s importantes de la computaci�n afectiva e introduce la tecnolog�a y la metodolog�a que se van a utilizar.
El cap�tulo 3 describe las herramientas utilizadas a lo largo del trabajo.
El cap�tulo 4 explica cada uno de los servicios web desarrollados durante el trabajo.

% Variable local para emacs, para  que encuentre el fichero maestro de
% compilaci�n y funcionen mejor algunas teclas r�pidas de AucTeX
%%%
%%% Local Variables:
%%% mode: latex
%%% TeX-master: "../Tesis.tex"
%%% End:
