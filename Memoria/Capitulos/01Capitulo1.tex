%---------------------------------------------------------------------
%
%                          Cap�tulo 1
%
%---------------------------------------------------------------------

\chapter{Introducci�n}

\begin{resumen}
	En este cap�tulo se va a realizar una introducci�n al trabajo que vamos a presentar.
	En primer lugar, se explicar� en qu� consiste el trabajo. A continuaci�n se expondr�n las principales dificultades a las que nos hemos enfrentado y las decisiones que se han tomado para afrontarlas. Por �ltimo, se presentar� la estructura del trabajo.

\end{resumen}

	El trabajo se centra en la detecci�n autom�tica del contenido emocional de un texto. Esto nos va a ayudar a evitar ambig�edades emocionales a la hora de interpretar un texto facilitando su lectura a personas con alg�n d�ficit en la percepci�n de emociones, como aquellas con Trastornos del Espectro Autista.
	
	Se implementar�n servicios web que permitan analizar emocionalmente un texto, identificando las emociones b�sicas que contiene y su intensidad, y hacer m�s explicito el significado emocional para hacer m�s f�cil su comprensi�n a las personas con alg�n tipo de discapacidad emocional.

	A lo largo de la realizaci�n del trabajo nos hemos ido encontrando una serie de problemas que se explican a continuaci�n.

	El cap�tulo 2 presenta los aspectos m�s importantes de la computaci�n afectiva e introduce a los servicios web, la metodolog�a Scrum y la integraci�n continua; metodolog�a y tecnolog�a que se va a utilizar.
	
	El cap�tulo 3 describe las herramientas utilizadas a lo largo del trabajo.
	
	El cap�tulo 4 explica cada uno de los servicios web desarrollados.
	
	El cap�tulo 5 se centra en el proceso de dise�o de la interfaz que tendr� la aplicaci�n web.
	
	El cap�tulo 6 describe el desarrollo de la aplicaci�n web que har� uso de los servicios web desarrollados.
	
	El cap�tulo 7 contiene la API de la aplicaci�n desarrollada.
	
	El cap�tulo 8 presenta los resultados obtenidos al finalizar el proyecto.
	
	El cap�tulo 9 contiene un resumen del trabajo realizado de manera individual por cada una de los componentes del proyecto.
	
	Finalmente, en el cap�tulo 10 explicamos de qu� manera hemos aplicado los conocimientos adquiridos a lo largo de la carrera.

% Variable local para emacs, para  que encuentre el fichero maestro de
% compilaci�n y funcionen mejor algunas teclas r�pidas de AucTeX
%%%
%%% Local Variables:
%%% mode: latex
%%% TeX-master: "../Tesis.tex"
%%% End:
