
%---------------------------------------------------------------------
%
%                          Capitulo 11
%
%---------------------------------------------------------------------
\setlength{\parskip}{10pt}
\chapter{Conclusiones y Trabajo futuro}

\begin{resumen}
En este capítulo se presenta el trabajo futuro que consideramos que podría implementarse si se continuara con el desarrollo de este proyecto. Junto con los objetivos que presentamos a continuación otros muchos que pudieran ir surgiendo.
\end{resumen}

\section{Conclusiones}
%--------
\label{cap11:sec:conclusiones}

\subsection{Motivación}
%--------
\label{cap11:sec:motivacion}

\subsection{Objetivo}
%--------
\label{cap11:sec:objetivo}

\subsection{¿Qué hemos realizado?}
%--------
\label{cap11:sec:que hemos realizado}

\subsection{Evaluación}
%--------
\label{cap11:sec:evaluacion}

\subsection{¿Qué hemos conseguido?}
%--------
\label{cap11:sec:que hemos conseguido}

\subsection{Conocimientos aplicados}
%--------
\label{cap11:sec:conocimientos aplicados}


\subsection{Conocimientos adquiridos}
%--------
\label{cap11:sec:conocimientos adquiridos}

\section{Trabajo futuro}
%--------
\label{cap11:sec:trabajo futuro}

Dada la magnitud de nuestro proyecto, muchos de los objetivos inicialmente propuestos no pudieron desarrollarse, así como muchos otros que fueron surgiendo a lo largo del desarrollo del mismo.
Por ello, con el fin de suplir las carencias del proyecto y dotarlo de mayor funcionalidad para que la aplicación sea más completa, consideramos que podemos dejar como trabajo futuro la implementación de estos objetivos que no llegamos a cubrir: \\



- \textbf{Añadir cambio de imágenes asociadas a las emociones}: al igual que nuestra aplicación permite el cambio de los colores asociados a cada emoción, se implementaría la posibilidad de realizar el cambio de imagen que también llevan asociadas las emociones. Para ello, deberíamos crear una sesión por cada usuario que utilizara nuestra aplicación con el fin de que las imágenes que seleccionase quedaran guardadas en su ordenador. \\


- \textbf{Implementación de detección automática de colores oscuros para las emociones}: tras la evaluación preliminar con usuarios, uno de ellos observó que, si seleccionaba un color oscuro para cualquiera de las emociones, las imágenes asociadas no se veían. Por tanto, consideramos un buen trabajo futuro, el hecho de que si seleccionamos un color oscuro, la imagen invierta su color de manera automática y de esta forma sea posible visualizarlas.\\ 



- \textbf{Añadir reconocimiento por voz}: se intentó realizar la implementación del reconocimiento por voz haciendo uso de la API de Google, Speech Recognition (la cual convierte el texto que escucha por el micrófono en texto escrito), pero como las peticiones que nosotras realizamos al servidor son HTTP y el explorador (Chrome) las reconoce como no seguras, no se permite el acceso al micrófono y tuvimos que abortar este objetivo. De ahí, surge este objetivo y dejarlo como trabajo futuro. \\



- \textbf{Cambiar tipo de peticiones al servidor}: las peticiones que actualmente se realizan al servidor son HTTP, lo que hace que se restrinja en cierta medida hacer uso de APIs externas (como por ejemplo la API de reconocimiento por voz), es por eso y por temas de seguridad por lo que quedaría justificado cambiar los tipos de peticiones a HTTPs.\\



- \textbf{Calcular de diferente manera los valores de los distintos tipos de frases}: el proyecto ha sido desarrollado para reconocer frases exclamativas, interrogativas y afirmativa. Como trabajo futuro, debería implementarse el reconocimiento de frases negativas, subordinadas, adjetivas... \\


- \textbf{Añadir posibilidad de insertar palabras que no aparecen en nuestro diccionario}: tras la evaluación preliminar con usuarios, se observó la necesidad de incluir en el diccionario términos más coloquiales. Por ello, sería útil la implementación de la inserción de nuevas palabras en nuestro diccionario. Una posible solución sería crear diferentes roles, con el fin de darle solo la oportunidad de inserción a aquel usuario que tuviera el rol de "tutor". Otra solución podría ser la creación de un buzón de sugerencia o similar, con el fin de que cuando una palabra fuera sugerida por un mínimo de usuarios con un grado aproximado para cada emoción ésta se insertara en el diccionario.\\


-\textbf{ Inserción de frases hechas y expresiones en el diccionario}:  un avance en el progreso de la aplicación sería la inserción en el diccionario de frases hechas y expresiones, con el fin de que el traductor fuera capaz de interpretar parte de la complejidad del castellano.\\



- \textbf{Implementación de una aplicación móvil}: actualmente nuestro proyecto es una aplicación web. Cada vez son más los usuarios que disponen de un dispositivo móvil, por ello también sería útil el desarrollo de una aplicación móvil.\\

- \textbf{Actualización de la interfaz para que el nivel de conformidad sea AAA} con el fin de que la web más accesible para personas con algún tipo de discapacidad.  \\




- \textbf{Creación de un log} para poder guardar los textos que se introducen en el diccionario, así como sus resultados para facilitar la depuración de éstos.\\


El principal factor que ha impedido realizar estas nuevas implementaciones ha sido el factor tiempo.  En un futuro próximo nos gustaría realizar el desarrollo de la aplicación de forma completa para que ésta sea una herramienta con la máxima funcionalidad posible y proporcione una ayuda precisa para todos los usuarios.\\


