%---------------------------------------------------------------------
%
%                          Capitulo 12
%
%---------------------------------------------------------------------
\setlength{\parskip}{10pt}
\chapter{Conclusions and Future work}

\begin{resumen}
This chapter presents the future work that we believe could be implemented if the development of this project were continued. Along with the objectives that we present below many others that could be emerging.
\end{resumen}

\section{Conclusions}
%--------
\label{cap12:sec:conclusiones}

\subsection{Motivation}
%--------
\label{cap12:sec:motivacion}

\subsection{Goals}
%--------
\label{cap12:sec:objetivo}

\subsection{What have we done?}
%--------
\label{cap12:sec:que hemos realizado}

\subsection{Evaluation}
%--------
\label{cap12:sec:evaluacion}

\subsection{What have we achieved?}
%--------
\label{cap12:sec:que hemos conseguido}

\subsection{Applied Knowledge}
%--------
\label{cap12:sec:conocimientos aplicados}


\subsection{Acquired knowledge}
%--------
\label{cap12:sec:conocimientos adquiridos}

\section{Future Work}
%--------
\label{cap12:sec:trabajo futuro}

Given the magnitude of our project, many of the initially proposed objectives could not be developed, as well as many others that were emerging during its development.
Therefore, in order to meet the shortcomings of the project and provide more functionality for the application to be more complete, we consider that we can leave as future work the implementation of these objectives that we do not cover:\\



- \textbf{Add change of images associated with emotions}: just as our application allows the change of colors associated with each emotion, the possibility of carrying out the image change that emotions are associated with would also be implemented. To do this, we should create a session for each user that uses our application so that the images you select will be saved on your computer.\\


- \textbf{Implementation of automatic detection of dark colors for emotions}: after the preliminary evaluation with users, one of them observed that, if he selected a dark color for any of the emotions, the associated images were not visible. Therefore, we consider a good future work, the fact that if we select a dark color, the image inverts its color automatically and in this way it is possible to visualize them.\\ 



- \textbf{Add recognition by voice}: the implementation of voice recognition was attempted using the Google API, Speech Recognition (which converts the text heard through the microphone into written text), but like the requests that we make to the server are HTTP and the browser (Chrome) recognizes them as not secure, access to the microphone is not allowed and we had to abort this goal. From there, this objective emerges and leave it as future work.\\



- \textbf{Change type of requests to the server}: the requests that are currently made to the server are HTTP, which makes it restrict to a certain extent to use external APIs (such as the voice recognition API), that's why and for security issues so it would be justified to change the types of requests to HTTPs.\\



- \textbf{Calculate in different ways the values of the different types of sentences}: the project has been developed to recognize exclamatory, interrogative and affirmative sentences. As future work, the recognition of negative phrases, subordinates, adjectives should be implemented ...\\


- \textbf{Add possibility of inserting words that do not appear in our dictionary}: after the preliminary evaluation with users, the need was observed to include more colloquial terms in the dictionary. Therefore, it would be useful to implement the insertion of new words in our dictionary. A possible solution would be to create different roles, in order to give only the opportunity of insertion to that user who had the role of "tutor". Another solution could be the creation of a suggestion box or similar, so that when a word is suggested by a minimum number of users with an approximate degree for each emotion it will be inserted in the dictionary.\\


-\textbf{Insertion of phrases and expressions in the dictionary}: an advance in the progress of the application would be the insertion in the dictionary of phrases and expressions, in order that the translator was able to interpret part of the complexity of Spanish.\\



- \textbf{Implementation of a mobile application}: our project is currently a web application. More and more users have a mobile device, so it would also be useful to develop a mobile application.\\

- \textbf{Update the interface so that the level of compliance is AAA} in order to make the web more accessible for people with some type of disability.\\



- \textbf{Creation of a log} to be able to save the texts that are introduced in the dictionary, as well as their results to facilitate the purification of these.\\


The main factor that has prevented these new implementations from being implemented has been the time factor. In the near future we would like to develop the application completely so that it is a tool with the maximum possible functionality and provide accurate help for all users.\\


