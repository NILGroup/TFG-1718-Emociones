%---------------------------------------------------------------------
%
%                          Capitulo 12
%
%---------------------------------------------------------------------
\setlength{\parskip}{10pt}
\chapter{Conclusions and Future work}

\begin{resumen}
This chapter presents the conclusions of the TFG and the future work that we believe could be implemented if the development of this project were continued.
\end{resumen}

\section{Conclusions}
%--------
\label{cap12:sec:conclusiones}

Nowadays, thanks to the Internet, we have quick and easy access to a huge amount of information. All we need is to interpret this information correctly to know if it is useful for us or not and how if so. At this point a problem arises because not everyone has the same easy to interpret a text, for example, the lack of emotional empathy in people with Autism Spectrum Disorders (ASD) impedes them to recognize the affective content in a text which can greatly alter its meaning. Something similar happens when these people want to write something on the net: the difficulties they have when expressing their emotions can cause misunderstandings. It is necessary to develop tools which could allow the digital integration of people with this type of disability.

The main goal of this project was to implement a series of web services that would automatically detect the degree of affective weight of a text in order to make it more accessible. The idea was to measure the degree of intensity for each of the five basic emotional categories with which we had decided to work: sadness, fear, joy, anger and disgust. In order to do it we had an affective dictionary which contained a series of words accompanied by its lexeme and five values from one to five (one for each emotional category). The dictionary would help us marking the words from the text with their emotional degrees. It will do the basis for the analysis. The emotional degrees of the words will allow us to obtain the final emotional degrees of the sentences of the text and, from these, we will obtain the definitive degrees for each emotional category for the text.

Once the services are developed, we will integrate them into an API to make them accessible to everyone who needs them and we will create a web application so our services will can be easily used by anyone. This website will allow an user to enter a text and it would display the results graphically, using emoticons and colors to make explicit the emotional categories contained in the text and marking the words that we consider emotional within it. In addition, emotional words (those whose grades have contributed to the result) will be marked within the text.

The web application has been submitted to two evaluations with end users and experts in which we have been able to test both its functionality and its interface. These evaluations showed us how useful our application can be as well as many aspects to improve. We corrected some of the problems detected and the rest of them have been documented for future work.

This project is an opportunity to apply all the knowledge acquired during the degree to a large project with real impact. Throughout the degree we had the opportunity to study a large series of subjects that have been very useful for the development of this project. Among all of them we can emphasize: \textbf{Foundations of the Programming}, \textbf{Technology of the Programming} and \textbf{Structure of Data and Algorithms} that helped us to acquire a structured and efficient way of thinking when programming, \textbf{Software Engineering}, which taught us to have the ability to manage a software project correctly and also the different software tests that can be done in a project, \textbf{Computer Audit} and \textbf{Evaluation of Configurations}, which have helped us improve the performance of our web services, \textbf{Web Applications}, \textbf{Extension of Databases} and \textbf{Web Engineering}, which gave us the necessary knowledge of HTML, CSS and Javascript so we could be able to develop our web interface, \textbf {Databases}, which helped us create the database that supports the model of Django and \textbf{Operating Systems}, \textbf{Networks}, \textbf{Networks and Security} and \textbf{Amplification of Networks and Operating Systems}, which helped us to carry out the management of the configuration of our Apache server as well as to carry out the project deployment on it. Finally, in the subject of \textbf{Ethics, Legislation and Profession} we learned everything necessary about licenses, both to protect our code and to know how to correctly use free software developed by third parties.

We have also learned many new things: configuration of a virtual host on an Apache server, python as programming language as well as making use of several of the libraries it has, such as Spacy and PyStemmer, Scrum methodology, through which we have improved our way of parallel work. We have also discovered tools that have facilitated the development of the project, such as, among others, Jenkins, a tool that performs tests to our project or Latex, a compiler that helped us to make this memoir.

\section{Future Work}
%--------
\label{cap12:sec:trabajo futuro}

Considering the magnitude of our project, many of the initially proposed goals could not be developed, as well as many others that emerged during its development.
Therefore, in order to meet the shortcomings of the project and make the application to be more complete, we consider that we can leave as future work the implementation of these goals that we could not cover:\\



- \textbf{Add the option to change the images associated with the emotional categories}: Our application gives the user the chance to change the color associated with each emotional categories. It would be also possible to give him the possibility of change the image associated to each category too. To do this, we should create a session for each user that uses our application so that the selected images would be saved on his computer.\\


- \textbf{Implementation of automatic detection of dark colors for emotional categories}: After the preliminary evaluation with users, one of them observed that, if a dark color was selected for any of the emotional categories, the associated images became invisible. Therefore, we consider a good future work, the fact that if a dark color is selected, the image inverts its color automatically so it could still be seen.\\ 



- \textbf{Add voice recognition}: The implementation of voice recognition was attempted using the Google API, Speech Recognition (which converts the text heard through the microphone into written text), but as the requests that we make to the server are HTTP and the browser (Chrome) recognizes them as not secure, access to the microphone is not allowed and we had to abort this goal. So, this goal has been left as future work.\\



- \textbf{Change the type of requests in the server}: The requests that are currently done to the server are using HTTP, which makes it restrict to a certain extent to use external APIs (such as the voice recognition API). Due that reason and, obviously, for security issues it would be justified to change the type of the requests to HTTPs.\\



- \textbf{Calculate in a different way the values of the different types of sentences}: The project has been developed to recognize exclamatory, interrogative and affirmative sentences. As future work, the recognition of negative sentences, subordinates and moreover should be implemented ...\\


- \textbf{Add the possibility of inserting words that do not appear in our dictionary}: After the preliminary evaluation with users, we could see the importance of including more colloquial terms in the dictionary. Therefore, it would be useful to implement the insertion of new words in our dictionary. A possible solution would be to create different roles, in order to give only the opportunity of insertion to the users who had the role of ``tutor''. Another possible solution could be the creation of a suggestion box or something similar, so when a word is suggested by a minimum number of users with an approximate degree for each emotion it will be inserted in the dictionary.\\


-\textbf{Insertion of sentences and expressions in the dictionary}: An important progress for the application would be the insertion in the dictionary of set phrases and expressions in order to make it able to interpret part of the complexity of Spanish.\\



- \textbf{Implementation of a mobile application}: Our project is currently a web application. More and more users have a mobile device, so it would also be useful to develop a mobile application.\\

- \textbf{Update the interface so that the level of compliance is AAA}: In order to make the web more accessible for people with some type of disability.\\



- \textbf{Creation of a log}: To be able to save the texts that are introduced in the dictionary, as well as their results to facilitate the depuration of these.\\


The main factor that has impeded these new implementations from being implemented has been time. In the near future we would like to develop the application completely so it could has the more functional it can be in order to provide accurate help for all users.\\


