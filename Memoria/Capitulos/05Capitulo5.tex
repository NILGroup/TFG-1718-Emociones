%---------------------------------------------------------------------
%
%                          Cap�tulo 5
%
%---------------------------------------------------------------------

\chapter{Dise�o de la aplicaci�n web}

\begin{resumen}
En este cap�tulo se explicar� el proceso seguido para dise�ar la interfaz que tendr� la aplicaci�n web que desarrollaremos.
Primero, en la secci�n 5.1, se presentan los prototipos que dise�amos cada una de las tres. En la secci�n 5.2, se muestra el prototipo que obtuvimos tras poner los tres en com�n. Finalmente, en la secci�n 5.3, se presenta la interfaz que finalmente tendr� la aplicaci�n.
\end{resumen}

%-------------------------------------------------------------------
\section{Prototipos iniciales}
%-------------------------------------------------------------------
\label{cap5:sec:prototipos_iniciales}

Cada una de nosotras dise�� un prototipo distinto para la aplicaci�n web. Los prototipos resultantes son bastante similares entre s�. A continuaci�n se muestran las principales secciones de la aplicaci�n, la pantalla principal y la pantalla que muestra el resultado del an�lisis, y sus diferencias en los tres prototipos.

\subsection{Pantalla principal}

La pantalla principal en todos los casos es bastante simple, un cuadro de texto donde introducir el texto a interpretar y un bot�n para ejecutar el interprete (Figura \ref{fig:inicial_paloma}). Puede incluir los emoticonos que representen las distintas emociones con las que trabajamos debajo del cuadro (Figura \ref{fig:inicial_gema}) o a la derecha de este (Figura \ref{fig:inicial_elena}).

\figura{Bitmap/Capitulo5/inicial_elena}{width=.6\textwidth}{fig:inicial_elena}{Pantalla inicial del prototipo de Elena}
\figura{Bitmap/Capitulo5/inicial_gema}{width=.6\textwidth}{fig:inicial_gema}{Pantalla inicial del prototipo de Gema}
\figura{Bitmap/Capitulo5/inicial_paloma}{width=.6\textwidth}{fig:inicial_paloma}{Pantalla inicial del prototipo de Paloma}

\subsection{Pantalla de an�lisis}

Los resultados se muestran de forma similar en los tres prototipos. Todos ellos utilizan emoticonos
\figura{Bitmap/Capitulo5/analisis_elena}{width=.6\textwidth}{fig:analisis_elena}{Pantalla de an�lisis del prototipo de Elena}
\figura{Bitmap/Capitulo5/analisis_gema}{width=.6\textwidth}{fig:analisis_gema}{Pantalla de an�lisis del prototipo de Gema}
\figura{Bitmap/Capitulo5/analisis_paloma}{width=.6\textwidth}{fig:analisis_paloma}{Pantalla de an�lisis del prototipo de Paloma}

%-------------------------------------------------------------------
\section{Prototipo com�n}
%-------------------------------------------------------------------
\label{cap5:sec:prototipo_comun}

Tras comparar los tres prototipos se lleg� a uno com�n.

\subsection{Pantalla principal}

\figura{Bitmap/Capitulo5/inicial1_final}{width=.6\textwidth}{fig:inicial1_final}{Pantalla inicial por defecto del prototipo final}
\figura{Bitmap/Capitulo5/configuracion}{width=.6\textwidth}{fig:configuracion}{Men� de opciones de la interfaz}
\figura{Bitmap/Capitulo5/inicial2_final}{width=.6\textwidth}{fig:inicial2_final}{Pantalla inicial alternativa del prototipo final}

\subsection{Pantalla de an�lisis}

\figura{Bitmap/Capitulo5/analisis1_final}{width=.6\textwidth}{fig:analisis1_final}{Pantalla de an�lisis con los porcentajes abajo}
\figura{Bitmap/Capitulo5/analisis2_final}{width=.6\textwidth}{fig:analisis2_final}{Pantalla de an�lisis con los porcentajes derecha}
\figura{Bitmap/Capitulo5/analisis3_final}{width=.6\textwidth}{fig:analisis3_final}{Pantalla de an�lisis con gr�fico}
\figura{Bitmap/Capitulo5/analisis4_final}{width=.6\textwidth}{fig:analisis4_final}{Pantalla de an�lisis mostrando las emociones que aparecen}
\figura{Bitmap/Capitulo5/analisis5_final}{width=.6\textwidth}{fig:analisis5_final}{Pantalla de an�lisis mostrando s�lo la mayoritaria}

%-------------------------------------------------------------------
\section{Interfaz final}
%-------------------------------------------------------------------
\label{cap5:sec:interfaz_final}

Tras reunirnos con la asociaci�n y teni�ndo en cuenta la informaci�n que nos proporcion� acordamos un prototipo final sobre el que realizar la interfaz de la aplicaci�n.
